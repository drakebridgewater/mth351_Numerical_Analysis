\documentclass{article}
\usepackage{enumerate}
\usepackage{fullpage}
\usepackage[fleqn]{amsmath}
\usepackage{amssymb}
\usepackage{graphicx}
\usepackage{hyperref}
\setlength{\parindent}{0pt} 
\newcommand{\myspace}{0.4cm}
\pagestyle{empty}
\usepackage{array}
\newcolumntype{C}[1]{>{\centering\let\newline\\\arraybackslash\hspace{0pt}}m{#1}}
\newcolumntype{L}[1]{>{\raggedright\let\newline\\\arraybackslash\hspace{0pt}}m{#1}}
\newcolumntype{R}[1]{>{\raggedleft\let\newline\\\arraybackslash\hspace{0pt}}m{#1}}
\DeclareMathOperator\erf{erf}

\begin{document}

\begin{center}

\large
\begin{tabular}{L{0.3\linewidth} C{0.3\linewidth} R{0.3\linewidth}}
\hline
Assignment 2	&MTH 351 -- Section 010		&Winter 2014 \\
\hline
\end{tabular}

\vspace{\myspace}

\end{center}

\begin{enumerate}

%%%QUESTION 1
\item {\bf [5 points]} Consider the following system of equations:

\begin{equation*}
\begin{bmatrix}
3		&-4\alpha \\
3\alpha	&-1\\
\end{bmatrix}  \begin{bmatrix}x_1 \\ x_2 \\ \end{bmatrix}= \begin{bmatrix}4 \\ 2 \\ \end{bmatrix},
\end{equation*}
where $\alpha$ is some real-valued parameter. For what value(s) of $\alpha$ does the system have:
\begin{enumerate} 
\item No solutions?
\item Infinitely many solutions?
\item A unique solution?
\end{enumerate}

\item {\bf[8 points]} Consider the system $Ax = b$, with 

\begin{tabular}{p{4cm} c p{3cm} p{3cm}}
$A = 
\begin{bmatrix}
 ~2     &2     &2 \\
 -1     &0     &1 \\
 ~3     &5     &6 \\
\end{bmatrix} $
&
&
&
$ b = \begin{bmatrix}-1 \\~3\\~0 \end{bmatrix}$
\\\\
\begin{tabular}{| c c c | c |}
 ~2     &2     &2	& -1\\
 -1     &0     &1 	& 3\\
 ~3     &5     &6 	& 0\\
\end{tabular}
& $\rightarrow$ 
&
\begin{tabular}{ l }
 $R_2 \Leftrightarrow R_3$      \\
 $R_2=R_2-2.5R_1$     \\
 $R_3=R_3-0.5R_2$     \\
\end{tabular}

&
\begin{tabular}{| c c c | c |}
 ~2     &2     &~2	& -1\\
 -2     &0     &~1 	& $\frac{5}{2}$ \\
 ~0     &0     &.5 	& $\frac{7}{4}$\\
\end{tabular}
\\\\
& $\rightarrow$ &
\begin{tabular}{ l }
      \\
 $R_2=R_2+2R_1$     \\
      \\
\end{tabular}
&
\begin{tabular}{| c c c | c |}
 ~2     &2     &~2	& -1\\
 ~0     &2     &~3 	& $\frac{1}{2}$ \\
 ~0     &0     &.5 	& $\frac{7}{4}$\\
\end{tabular}
\\\\\\ \hline	 \\

$P = 
\begin{bmatrix}
 ~1     &0     &0 \\
 ~0     &0     &1 \\
 ~0     &1     &0 \\
\end{bmatrix} $
& 
$L = 
\begin{bmatrix}
 ~0     &0     &0 \\
 ~0     &0     &0 \\
 ~0     &0     &0 \\
\end{bmatrix} $
&
$U = 
\begin{bmatrix}
 ~0     &0     &0 \\
 ~0     &0     &0 \\
 ~0     &0     &0 \\
\end{bmatrix} $
&
\begin{tabular}{| c c c | c |}
 ~2     &2     &~2	& -1\\
 ~0     &2     &~3 	& $\frac{1}{2}$ \\
 ~0     &0     &.5 	& $\frac{7}{4}$\\
\end{tabular}

\end{tabular}



\begin{enumerate}
\item Compute the $LU$ factorization of $A$ by hand, and use it to solve the system as shown in class.
\item Compute the determinant of $A$ using the usual method (see \href{http://en.wikipedia.org/wiki/Determinant#3.C2.A0.C3.97.C2.A03_matrices}{here} if you do not recall how to do this).
\item Explain how the determinant of any $n \times n$ matrix $A$ can be easily computed from the $LU$ factorization of $A$, using only $n-1$ multiplications. Use your results from parts (a) and (b) to verify that this works for the matrix $A$ of this problem.

Hint: you may use the following result from linear algebra: $\det(AB)$ = $\det(A) \cdot \det(B)$.
\end{enumerate}

\item {\bf [7 points]} For certain special types of matrix, the $LU$ factorization has a simpler form, which can be computed more efficiently. A {\em positive definite} matrix is a symmetric, $n\times n$ matrix $A$ such that the product $x^TAx$ is positive for any non-zero vector $x \in \mathbb{R}^n$. It can be shown that the matrix
\begin{equation*}
A = \begin{bmatrix}
     ~1    &-2     &~1 \\
    -2     &~8     &~6 \\
     ~1     &~6    &19 \\
\end{bmatrix}
\end{equation*}
is positive definite.
\begin{enumerate}
\item Compute the $LU$ factorization of $A$ by hand.
\item You may notice that every {\bf row} of $U$ is a scalar multiple of the corresponding {\bf column} in $L$. Thus, show that you can write the $LU$ factorization of $A$ as
\begin{equation*}
A = L D L^T,
\end{equation*}
where $L$ is lower triangular and $D$ is a diagonal matrix. This can be done for any symmetric matrix (even if it is not positive definite).
\item As a consequence of the matrix being positive definite, all the elements in $D$ should be positive. Show that you can therefore write the $LU$ factorization of $A$ as
\begin{equation*}
A = LL^T
\end{equation*}
where $L$ is a lower triangular matrix (it is not exactly the same lower triangular matrix as you get from the $LU$ factorization). This is known as the Cholesky factorization, and it can be computed for any positive definite matrix.
\end{enumerate}

\end{enumerate}
\end{document}

